\documentclass[12pt,a4paper]{article}
\usepackage[utf8]{inputenc}
\usepackage[english]{babel}
\usepackage{graphicx}
\usepackage{amsmath}
\usepackage{booktabs}
\usepackage{float}
\usepackage{geometry}
\usepackage{caption}
\usepackage{subcaption}
\usepackage{longtable}
\usepackage{array}
\usepackage{url}

\geometry{margin=2.5cm}

% Set image path
\graphicspath{{../images/},{../PCB/Photos/},{../Schematic/}}

\begin{document}

% ============================================
% CUSTOM COVER PAGE
% ============================================
\begin{titlepage}
    \centering
    \vspace*{2cm}

    {\large Electrical Engineering}\\[3cm]

    % Title
    {\huge \textbf{PolySense Station}}\\[0.5cm]
    {\Large \textbf{Environmental Monitoring System}}\\[3cm]

    % Subtitle
    {\large Multi-sensor data collection, temporal analysis, and machine learning}\\[0.3cm]
    {\large applied to climate monitoring}\\[4cm]

    \vfill

    % Author information
    \begin{flushleft}
        \large
        \textbf{Author:} Jonas Souza\\
        \textbf{Education:} Electrical Engineer\\
        \textbf{Date:} \today
    \end{flushleft}

    \vspace{1cm}

\end{titlepage}

\newpage

\tableofcontents
\newpage

\section{Executive Summary}

This report presents a comprehensive analysis of the \textbf{PolySense Station} project, an autonomous environmental data acquisition and analysis system developed from scratch. The system combines custom hardware with 7 integrated sensors, a Raspberry Pi Pico microcontroller, and a complete data analysis pipeline with 13 specialized notebooks.

\subsection{Key Achievements}

\begin{itemize}
    \item \textbf{Real data collection:} 82,430 measurements over 30 days (September 2025)
    \item \textbf{Temporal resolution:} 30-second sampling interval
    \item \textbf{Sensor redundancy:} 7 temperature sensors, 2 humidity sensors, 2 pressure sensors
    \item \textbf{Validation against INMET station:} National Institute of Meteorology
    \item \textbf{LSTM model accuracy:} MAE < 1°C for 1-hour ahead temperature prediction
    \item \textbf{Climate clustering:} 4 distinct regimes identified via GMM
    \item \textbf{Public dataset:} Available on Kaggle with 82,430 records
\end{itemize}

\section{Introduction}

\subsection{Context}

Environmental monitoring is essential for understanding local climate patterns, validating predictive models, and developing climate-sensitive automation systems. This project implements a complete weather station with automatic data collection, SD card storage, and advanced data analysis.

\subsection{Objectives}

\begin{enumerate}
    \item Develop data acquisition hardware with sensor redundancy
    \item Collect environmental data (30-second interval)
    \item Validate measurements against official meteorological station (INMET)
    \item Perform comprehensive exploratory data analysis
    \item Apply machine learning techniques for pattern identification
    \item Develop temperature predictive models using LSTM neural networks
    \item Apply digital signal processing for noise reduction
\end{enumerate}

\subsection{Location and Period}

\begin{itemize}
    \item \textbf{Location:} Vitória da Conquista, Bahia, Brazil
    \item \textbf{Altitude:} 923 meters
    \item \textbf{Collection period:} August 31 to September 30, 2025
    \item \textbf{Total records:} 82,430 measurements
    \item \textbf{Dataset size:} 4.8 MB (raw CSV)
\end{itemize}

\section{Data Collection Methodology}

\subsection{Hardware Architecture}

The system was developed using the \textbf{Raspberry Pi Pico} (RP2040) microcontroller programmed in MicroPython, with the following interfaces:

\begin{itemize}
    \item \textbf{Dual I2C:} Two I2C buses to minimize electrical interference
    \item \textbf{SPI:} Interface for SD card writing
    \item \textbf{OneWire:} Protocol for DS18B20 temperature sensor
    \item \textbf{Analog:} ADC converter for NTC thermistor
    \item \textbf{OLED Display:} 128x64 screen for real-time monitoring
    \item \textbf{RTC:} Real-time clock for precise timestamps
\end{itemize}

\textbf{Hardware Design:} The system includes a custom PCB for sensor integration and a detailed schematic for replication:

\begin{figure}[H]
    \centering
    \includegraphics[width=0.95\textwidth]{PCB_front.jpg}
    \caption{Photograph of the PolySense Station PCB (Printed Circuit Board)}
    \label{fig:pcb_front}
\end{figure}

\begin{figure}[H]
    \centering
    \includegraphics[width=0.95\textwidth]{Schematic_Protoboard.png}
    \caption{Protoboard circuit schematic of the PolySense Station}
    \label{fig:schematic_protoboard}
\end{figure}

\begin{figure}[H]
    \centering
    \includegraphics[width=0.95\textwidth]{Schematic_Sensor.png}
    \caption{Detailed schematic of integrated sensors}
    \label{fig:schematic_sensor}
\end{figure}

\subsection{Implemented Sensor Suite}

\begin{table}[H]
\centering
\caption{Sensors integrated into PolySense Station}
\scriptsize
\begin{tabular}{@{}llll@{}}
\toprule
\textbf{Sensor} & \textbf{Type} & \textbf{Measurements} & \textbf{Protocol} \\ \midrule
MPU6050 & Gyroscope/Accelerometer & Temperature & I2C \\
AHT20 & Environmental & Temperature, Humidity & I2C \\
BMP280 & Barometric & Temperature, Pressure & I2C \\
BMP180 & Barometric & Temperature, Pressure & I2C \\
DS18B20 & OneWire Digital & Temperature & OneWire \\
NTC (KY-028) & Analog Thermistor & Temperature & ADC \\
DHT11 & Environmental & Temperature, Humidity & Digital \\ \bottomrule
\end{tabular}
\label{tab:sensors}
\end{table}

\textbf{Redundancy justification:}
\begin{itemize}
    \item Cross-validation between multiple sensors
    \item Detection of instrumental failures
    \item Reference sensors (DS18B20, BMP280, AHT20)
\end{itemize}

\subsection{Acquisition Firmware}

The main firmware (\texttt{main.py}, 226 lines) implements:

\begin{itemize}
    \item Synchronous reading of all sensors every 30 seconds
    \item Writing to SD card with RTC timestamp
    \item Rotating OLED display with 3 information screens
    \item Feedback LED (blinks on successful write)
    \item Safe ejection system via button
    \item Error handling and write status
\end{itemize}

\textbf{CSV storage format:}
\begin{verbatim}
Timestamp, Temp_MPU6050_C, Temp_AHT20_C, Umid_AHT20_pct,
Temp_BMP280_C, Press_BMP280_hPa, Temp_BMP180_C,
Press_BMP180_hPa, Temp_DS18B20_C, Temp_NTC_C,
Temp_DHT11_C, Umid_DHT11_pct
\end{verbatim}

\begin{figure}[H]
    \centering
    \includegraphics[width=0.95\textwidth]{Serial_output.png}
    \caption{Serial output of the firmware during PolySense Station operation}
    \label{fig:serial_output}
\end{figure}

\subsection{Validation Against INMET Station}

Data was validated against the official INMET meteorological station:
\begin{itemize}
    \item \textbf{Validation dataset:} 719 records from INMET station
    \item \textbf{INMET variables:} Temperature, humidity, pressure, wind, radiation, precipitation
    \item \textbf{Combined dataset:} 331 hourly aggregated records
\end{itemize}

\section{Module 1: Exploratory Data Analysis (Notebooks 01-05)}

\subsection{Notebook 01: Initial Exploratory Analysis}

\subsubsection{Temperature Distributions}

Analysis of temperature distributions revealed that all 7 sensors show consistent patterns:

\begin{figure}[H]
    \centering
    \includegraphics[width=0.95\textwidth]{data_analysis/01_Temperature_Data_Histogram_in_September_Vitória_da_Conquista.png}
    \caption{Histogram of temperature distributions (7 sensors)}
    \label{fig:temp_hist}
\end{figure}

\textbf{Observations:}
\begin{itemize}
    \item Approximately normal distributions with peaks between 20-22°C
    \item BMP280, AHT20, DS18B20 sensors with smooth curves
    \item DHT11 exhibits quantization artifacts (discrete steps)
    \item NTC thermistor shows greater dispersion (sensitivity to solar radiation)
\end{itemize}

\begin{figure}[H]
    \centering
    \includegraphics[width=0.95\textwidth]{data_analysis/01_Comparison_of_Temperature_Sensor_Distributions_September_Vitória_da_Conquista.png}
    \caption{Comparative boxplot - Temperature distribution between sensors}
    \label{fig:temp_boxplot}
\end{figure}

\textbf{Main findings:}
\begin{itemize}
    \item Excellent agreement between sensors (very close medians)
    \item Upper outliers identify daytime temperature peaks
    \item Consistent interquartile range between sensors
    \item DHT11 shows greater variability (lower resolution)
\end{itemize}

\subsubsection{Humidity Distributions}

\begin{figure}[H]
    \centering
    \includegraphics[width=0.95\textwidth]{data_analysis/01_Humidity_Data_Histogram_in_September_Vitória_da_Conquista.png}
    \caption{Histogram of relative humidity (AHT20 and DHT11)}
    \label{fig:humid_hist}
\end{figure}

\textbf{Identified bimodal pattern:}
\begin{itemize}
    \item First peak: 50-55\% (daytime period)
    \item Second peak: 75-80\% (nighttime period with fog formation)
    \item Consistent with September climate on the plateau (altitude 923m)
\end{itemize}

\begin{figure}[H]
    \centering
    \includegraphics[width=0.95\textwidth]{data_analysis/01_Comparison_of_Humidity_Sensor_Distributions_September_Vitória_da_Conquista.png}
    \caption{Comparative boxplot of humidity (AHT20 vs DHT11)}
    \label{fig:humid_boxplot}
\end{figure}

\subsubsection{Barometric Pressure Distributions}

\begin{figure}[H]
    \centering
    \includegraphics[width=0.95\textwidth]{data_analysis/01_Pressure_Data_Histogram_in_September_Vitória_da_Conquista.png}
    \caption{Histogram of barometric pressure (BMP280 and BMP180)}
    \label{fig:press_hist}
\end{figure}

\textbf{Observed characteristics:}
\begin{itemize}
    \item Range: 914-924 hPa (mean ~920 hPa)
    \item Distribution consistent with 923m altitude
    \item Excellent agreement between BMP280 and BMP180
\end{itemize}

\begin{figure}[H]
    \centering
    \includegraphics[width=0.95\textwidth]{data_analysis/01_Comparison_of_Pressure_Sensor_Distributions_September_Vitória_da_Conquista.png}
    \caption{Comparative boxplot of pressure (BMP280 vs BMP180)}
    \label{fig:press_boxplot}
\end{figure}

\subsection{Notebook 02: Correlation Analysis}

\subsubsection{Global Correlation Matrix}

\begin{figure}[H]
    \centering
    \includegraphics[width=0.95\textwidth]{data_analysis/02_Correlation_Matrix_All_Environmental_SensorsnVitória_da_Conquista_September_2025.png}
    \caption{Correlation matrix between all 11 measured variables}
    \label{fig:corr_matrix}
\end{figure}

\textbf{Identified correlations:}
\begin{itemize}
    \item \textbf{Temperature sensors (among themselves):} r > 0.99
    \item \textbf{Humidity sensors (AHT20 vs DHT11):} r = 0.985
    \item \textbf{Pressure sensors (BMP280 vs BMP180):} r = 0.984
    \item \textbf{Temperature vs Humidity:} r = -0.868 (strong inverse)
    \item \textbf{Temperature vs Pressure:} r = -0.559 (moderate inverse)
    \item \textbf{Humidity vs Pressure:} r = 0.463 (weak positive)
\end{itemize}

\subsubsection{Bivariate Relationships with KDE}

\begin{figure}[H]
    \centering
    \includegraphics[width=0.95\textwidth]{data_analysis/02_Relationship_between_Temperature_BMP280_and_Humidity_AHT20nVitória_da_Conquista_September_2025.png}
    \caption{Temperature vs Humidity relationship with probability density (KDE)}
    \label{fig:temp_humid_kde}
\end{figure}

\textbf{Physical interpretation:}
\begin{itemize}
    \item Strong negative correlation (r = -0.868)
    \item Daytime warming reduces relative humidity
    \item Nighttime cooling increases humidity (fog formation)
\end{itemize}

\begin{figure}[H]
    \centering
    \includegraphics[width=0.95\textwidth]{data_analysis/02_Relationship_between_Temperature_BMP280_and_Pressure_BMP180nVitória_da_Conquista_September_2025.png}
    \caption{Temperature vs Barometric Pressure relationship}
    \label{fig:temp_press_kde}
\end{figure}

\begin{figure}[H]
    \centering
    \includegraphics[width=0.95\textwidth]{data_analysis/02_Relationship_between_Humidity_AHT20_and_Pressure_BMP180nVitória_da_Conquista_September_2025.png}
    \caption{Humidity vs Barometric Pressure relationship}
    \label{fig:humid_press_kde}
\end{figure}

\subsection{Notebook 03: Missing Data Analysis}

\begin{figure}[H]
    \centering
    \includegraphics[width=0.9\textwidth]{data_analysis/03_Data_Completeness_per_Temperature_Sensor_Vitória_da_Conquista_September_2025.png}
    \caption{Data completeness - Temperature sensors}
    \label{fig:completeness_temp}
\end{figure}

\begin{figure}[H]
    \centering
    \includegraphics[width=0.9\textwidth]{data_analysis/03_Data_Completeness_per_Humidity_Sensor_Vitória_da_Conquista_September_2025.png}
    \caption{Data completeness - Humidity sensors}
    \label{fig:completeness_humid}
\end{figure}

\begin{figure}[H]
    \centering
    \includegraphics[width=0.9\textwidth]{data_analysis/03_Data_Completeness_per_Pressure_Sensor_Vitória_da_Conquista_September_2025.png}
    \caption{Data completeness - Pressure sensors}
    \label{fig:completeness_press}
\end{figure}

\textbf{Quality results:}
\begin{itemize}
    \item DS18B20: Highest completeness (>99.99\%)
    \item DHT11: Minor gaps (~99.98\%)
    \item All sensors: >99.9\% data available
    \item High reliability of the acquisition system
\end{itemize}

\subsection{Notebook 04: Sensor Validation}

\subsubsection{Bland-Altman Analysis - Temperature}

\begin{figure}[H]
    \centering
    \includegraphics[width=0.95\textwidth]{data_analysis/04_BLAND_ALTMAN_ANALYSIS_Temperature_Sensors_Pairwise_Comparison.png}
    \caption{Bland-Altman: Comparison between pairs of temperature sensors}
    \label{fig:bland_altman_temp}
\end{figure}

\textbf{Inter-sensor validation:}
\begin{itemize}
    \item Mean bias < 0.5°C between sensors
    \item Limits of agreement within ±2°C
    \item Excellent agreement between BMP280, AHT20, and DS18B20
\end{itemize}

\begin{figure}[H]
    \centering
    \includegraphics[width=0.95\textwidth]{data_analysis/04_BLAND_ALTMAN_ANALYSIS_Sensors_vs_Reference_Temp_Ins_C.png}
    \caption{Bland-Altman: Sensors vs INMET Reference}
    \label{fig:bland_altman_inmet}
\end{figure}

\textbf{Validation against INMET:}
\begin{itemize}
    \item Mean bias < 2°C compared to the official station
    \item NTC thermistor shows systematic offset (solar radiation)
    \item Digital sensors validate reliably
\end{itemize}

\subsubsection{Bland-Altman Analysis - Humidity and Pressure}

\begin{figure}[H]
    \centering
    \includegraphics[width=0.95\textwidth]{data_analysis/04_BLAND_ALTMAN_ANALYSIS_Humidity_Sensors_Pairwise_Comparison.png}
    \caption{Bland-Altman: Comparison of humidity sensors (AHT20 vs DHT11)}
    \label{fig:bland_altman_humid}
\end{figure}

\begin{figure}[H]
    \centering
    \includegraphics[width=0.95\textwidth]{data_analysis/04_BLAND_ALTMAN_ANALYSIS_Pressure_Sensors_BMP280_vs_BMP180.png}
    \caption{Bland-Altman: Comparison of pressure sensors}
    \label{fig:bland_altman_press}
\end{figure}

\subsection{Notebook 05: Temporal Analysis}

\subsubsection{Complete Time Series}

\begin{figure}[H]
    \centering
    \includegraphics[width=0.95\textwidth]{data_analysis/05_Temperature_Sensors_Comparison_September_2025_Vitória_da_Conquista.png}
    \caption{Complete temperature time series (September 2025)}
    \label{fig:temp_ts}
\end{figure}

\textbf{Identified patterns:}
\begin{itemize}
    \item Clear diurnal cycle with 10-12°C amplitude
    \item Temperature peak: 14:00-16:00 local time
    \item Temperature minimum: 06:00-07:00 (before sunrise)
    \item Weekly variations associated with synoptic systems
\end{itemize}

\begin{figure}[H]
    \centering
    \includegraphics[width=0.95\textwidth]{data_analysis/05_Humidity_Sensors_Comparison_September_2025_Vitória_da_Conquista.png}
    \caption{Humidity time series (September 2025)}
    \label{fig:humid_ts}
\end{figure}

\begin{figure}[H]
    \centering
    \includegraphics[width=0.95\textwidth]{data_analysis/05_Pressure_Sensors_Comparison_September_2025_Vitória_da_Conquista.png}
    \caption{Barometric pressure time series (September 2025)}
    \label{fig:press_ts}
\end{figure}

\textbf{Pressure observations:}
\begin{itemize}
    \item Oscillations of 10-12 hPa throughout the month
    \item Variations follow frontal passages
    \item Average pressure of 920 hPa (consistent with altitude)
\end{itemize}

\subsection{Notebook 06: Time Series Decomposition}

\begin{figure}[H]
    \centering
    \includegraphics[width=0.95\textwidth]{data_analysis/06_Time_Series_Decomposition_plot_01.png}
    \caption{Time series decomposition - Temperature}
    \label{fig:decomp_temp}
\end{figure}

\textbf{Identified components:}
\begin{itemize}
    \item \textbf{Trend:} Slight warming until mid-September, then cooling
    \item \textbf{Seasonal:} Strong 24-hour periodicity
    \item \textbf{Residuals:} Clean with few anomalies
\end{itemize}

\begin{figure}[H]
    \centering
    \includegraphics[width=0.95\textwidth]{data_analysis/06_Time_Series_Decomposition_plot_03.png}
    \caption{Time series decomposition - Humidity}
    \label{fig:decomp_humid}
\end{figure}

\begin{figure}[H]
    \centering
    \includegraphics[width=0.95\textwidth]{data_analysis/06_Time_Series_Decomposition_plot_05.png}
    \caption{Time series decomposition - Pressure}
    \label{fig:decomp_press}
\end{figure}

\section{Module 2: Machine Learning (Notebooks 07-11)}

\subsection{Notebook 07: Anomaly Detection}

\subsubsection{Isolation Forest}

\begin{figure}[H]
    \centering
    \includegraphics[width=0.95\textwidth]{machine_learning/07_Isolation_Forest_Anomaly_Detection_Across_All_Sensors.png}
    \caption{Anomaly detection via Isolation Forest (all sensors)}
    \label{fig:isolation_forest}
\end{figure}

\textbf{Results:}
\begin{itemize}
    \item \textbf{3.2\% of measurements} classified as anomalies
    \item Anomalies concentrated in specific time windows
    \item Successful isolation of instrumental failures
\end{itemize}

\begin{figure}[H]
    \centering
    \includegraphics[width=0.95\textwidth]{machine_learning/07_Heatmap_Sensor_Deviations_in_Detected_AnomaliesnRed_High_deviation_Green_Normal_showing_50_random_anomalies.png}
    \caption{Heatmap of sensor deviations in detected anomalies}
    \label{fig:anomaly_heatmap}
\end{figure}

\textbf{Deviation analysis:}
\begin{itemize}
    \item Red: High divergence from the mean
    \item Green: Normal behavior
    \item Identification of which sensors deviated during anomalies
\end{itemize}

\subsubsection{Frozen Value Detection}

\begin{figure}[H]
    \centering
    \includegraphics[width=0.95\textwidth]{machine_learning/07_Stuck_Values_Detection_Δ_001C.png}
    \caption{Detection of frozen values (sensor freezes, $\Delta < 0.01$°C)}
    \label{fig:stuck_values}
\end{figure}

\textbf{Detection criteria:}
\begin{itemize}
    \item Identification of measurements where $\Delta < 0.01$°C for >10 minutes
    \item Detection of temporary sensor failures
    \item Validation of the need for multi-sensor redundancy
\end{itemize}

\subsection{Notebook 08: Decision Tree Regression}

\begin{figure}[H]
    \centering
    \includegraphics[width=0.95\textwidth]{machine_learning/08_Decision_Tree_Temperature_Prediction_All_Sensors_Average_September_2025_Analysis_in_Vitória_da_Conquista.png}
    \caption{Decision tree structure for temperature prediction}
    \label{fig:decision_tree}
\end{figure}

\begin{figure}[H]
    \centering
    \includegraphics[width=0.9\textwidth]{machine_learning/08_Which_Variable_Most_Influences_the_Analysis_Vitória_da_Conquista_September_2025.png}
    \caption{Feature importance for temperature prediction}
    \label{fig:feature_importance}
\end{figure}

\textbf{Most important features:}
\begin{itemize}
    \item AHT20 and BMP280 among the main predictors
    \item Humidity has strong predictive power for temperature
    \item Model presents interpretable rules
\end{itemize}

\subsection{Notebook 09: Clustering with Gaussian Mixture Model (GMM)}

\begin{figure}[H]
    \centering
    \includegraphics[width=0.95\textwidth]{machine_learning/09_Gaussian_Mixture_Clustering_plot_01.png}
    \caption{GMM Clustering - 2D visualization (K=4 climate regimes)}
    \label{fig:gmm_2d}
\end{figure}

\begin{figure}[H]
    \centering
    \includegraphics[width=0.95\textwidth]{machine_learning/09_3D_Climate_Clustering_GMM_K4nVitória_da_Conquista_September_2025.png}
    \caption{GMM Clustering - 3D visualization (Temperature, Humidity, Pressure)}
    \label{fig:gmm_3d}
\end{figure}

\textbf{4 identified climate regimes:}
\begin{enumerate}
    \item \textbf{Cluster 1 (Blue):} Cold-Humid (nighttime with fog)
    \item \textbf{Cluster 2 (Orange):} Hot-Dry (afternoon with maximums)
    \item \textbf{Cluster 3 (Green):} Morning transition (progressive warming)
    \item \textbf{Cluster 4 (Red):} Evening transition (cooling)
\end{enumerate}

\subsection{Notebook 10: Clustering with KMeans}

\begin{figure}[H]
    \centering
    \includegraphics[width=0.95\textwidth]{machine_learning/10_KMeans_Clustering_plot_01.png}
    \caption{KMeans clustering - 2D visualization (K=4)}
    \label{fig:kmeans_2d}
\end{figure}

\begin{figure}[H]
    \centering
    \includegraphics[width=0.95\textwidth]{machine_learning/10_3D_Climate_Clustering_K_Means_K4nVitória_da_Conquista_September_2025.png}
    \caption{KMeans clustering - 3D visualization}
    \label{fig:kmeans_3d}
\end{figure}

\textbf{Consistency between methods:}
\begin{itemize}
    \item KMeans validates the 4 climate states identified by GMM
    \item Well-separated clusters in three-dimensional space
    \item Confirmation of climate patterns throughout September
\end{itemize}

\subsection{Notebook 11: Prediction with LSTM Neural Networks}

\begin{figure}[H]
    \centering
    \includegraphics[width=0.9\textwidth]{machine_learning/11_Error_Evolution_During_Training.png}
    \caption{Error evolution during LSTM model training}
    \label{fig:lstm_loss}
\end{figure}

\textbf{Model characteristics:}
\begin{itemize}
    \item LSTM architecture with 30-minute time window
    \item Convergence observed after ~50 epochs
    \item Consistent reduction of training and validation error
\end{itemize}

\begin{figure}[H]
    \centering
    \includegraphics[width=0.95\textwidth]{machine_learning/11_Real_Temperature_vs_Model_Prediction_Vitória_da_Conquista_in_the_Last_Week_of_September_2025.png}
    \caption{Real Temperature vs LSTM Predictions (last week of September 2025)}
    \label{fig:lstm_predictions}
\end{figure}

\textbf{Model performance:}
\begin{itemize}
    \item \textbf{MAE < 1°C} for 1-hour ahead prediction
    \item Model accurately captures diurnal patterns
    \item Predictions fit real temperatures in the validation period
    \item Temporal dependencies effectively captured
\end{itemize}

\section{Module 3: Digital Signal Processing (Notebooks 12-13)}

\subsection{Notebook 12: Digital Filters}

\subsubsection{Filter Comparison - Temperature}

\begin{figure}[H]
    \centering
    \includegraphics[width=0.95\textwidth]{signal_processing/12_Temperature_Analysis_Panel_Complete_View_Raw_vs_FilterednVitoria_da_Conquista_September_2025_1.png}
    \caption{Comparison of digital filters - Temperature Sensor 1 (Moving Average)}
    \label{fig:temp_filter_ma_1}
\end{figure}

\begin{figure}[H]
    \centering
    \includegraphics[width=0.95\textwidth]{signal_processing/12_Temperature_Analysis_Panel_Complete_View_Raw_vs_FilterednVitoria_da_Conquista_September_2025_2.png}
    \caption{Comparison of digital filters - Temperature Sensor 2 (Moving Average)}
    \label{fig:temp_filter_ma_2}
\end{figure}

\begin{figure}[H]
    \centering
    \includegraphics[width=0.95\textwidth]{signal_processing/12_Temperature_Analysis_Panel_Complete_View_Raw_vs_FilterednVitoria_da_Conquista_September_2025_3.png}
    \caption{Comparison of digital filters - Temperature Sensor 3 (Moving Average)}
    \label{fig:temp_filter_ma_3}
\end{figure}

\begin{figure}[H]
    \centering
    \includegraphics[width=0.95\textwidth]{signal_processing/12_Temperature_Analysis_Panel_Complete_View_Raw_vs_FilterednVitoria_da_Conquista_September_2025_4.png}
    \caption{Comparison of digital filters - Temperature Sensor 4 (Moving Average)}
    \label{fig:temp_filter_ma_4}
\end{figure}

\begin{figure}[H]
    \centering
    \includegraphics[width=0.95\textwidth]{signal_processing/12_Temperature_Analysis_Panel_Complete_View_Raw_vs_FilterednVitoria_da_Conquista_September_2025_5.png}
    \caption{Comparison of digital filters - Temperature Sensor 5 (Moving Average)}
    \label{fig:temp_filter_ma_5}
\end{figure}

\begin{figure}[H]
    \centering
    \includegraphics[width=0.95\textwidth]{signal_processing/12_Temperature_Analysis_Panel_Complete_View_Raw_vs_FilterednVitoria_da_Conquista_September_2025_6.png}
    \caption{Comparison of digital filters - Temperature Sensor 6 (Moving Average)}
    \label{fig:temp_filter_ma_6}
\end{figure}

\begin{figure}[H]
    \centering
    \includegraphics[width=0.95\textwidth]{signal_processing/12_Temperature_Analysis_Panel_Complete_View_Raw_vs_FilterednVitoria_da_Conquista_September_2025_7.png}
    \caption{Comparison of digital filters - Temperature Sensor 7 (Moving Average)}
    \label{fig:temp_filter_ma_7}
\end{figure}

\begin{figure}[H]
    \centering
    \includegraphics[width=0.95\textwidth]{signal_processing/12_Temperature_Analysis_Panel_Complete_View_Raw_vs_Median_FilterednVitoria_da_Conquista_September_2025_1.png}
    \caption{Comparison of digital filters - Temperature Sensor 1 (Median)}
    \label{fig:temp_filter_median_1}
\end{figure}

\begin{figure}[H]
    \centering
    \includegraphics[width=0.95\textwidth]{signal_processing/12_Temperature_Analysis_Panel_Complete_View_Raw_vs_Median_FilterednVitoria_da_Conquista_September_2025_2.png}
    \caption{Comparison of digital filters - Temperature Sensor 2 (Median)}
    \label{fig:temp_filter_median_2}
\end{figure}

\begin{figure}[H]
    \centering
    \includegraphics[width=0.95\textwidth]{signal_processing/12_Temperature_Analysis_Panel_Complete_View_Raw_vs_Median_FilterednVitoria_da_Conquista_September_2025_3.png}
    \caption{Comparison of digital filters - Temperature Sensor 3 (Median)}
    \label{fig:temp_filter_median_3}
\end{figure}

\begin{figure}[H]
    \centering
    \includegraphics[width=0.95\textwidth]{signal_processing/12_Temperature_Analysis_Panel_Complete_View_Raw_vs_Median_FilterednVitoria_da_Conquista_September_2025_4.png}
    \caption{Comparison of digital filters - Temperature Sensor 4 (Median)}
    \label{fig:temp_filter_median_4}
\end{figure}

\begin{figure}[H]
    \centering
    \includegraphics[width=0.95\textwidth]{signal_processing/12_Temperature_Analysis_Panel_Complete_View_Raw_vs_Median_FilterednVitoria_da_Conquista_September_2025_5.png}
    \caption{Comparison of digital filters - Temperature Sensor 5 (Median)}
    \label{fig:temp_filter_median_5}
\end{figure}

\begin{figure}[H]
    \centering
    \includegraphics[width=0.95\textwidth]{signal_processing/12_Temperature_Analysis_Panel_Complete_View_Raw_vs_Median_FilterednVitoria_da_Conquista_September_2025_6.png}
    \caption{Comparison of digital filters - Temperature Sensor 6 (Median)}
    \label{fig:temp_filter_median_6}
\end{figure}

\begin{figure}[H]
    \centering
    \includegraphics[width=0.95\textwidth]{signal_processing/12_Temperature_Analysis_Panel_Complete_View_Raw_vs_Median_FilterednVitoria_da_Conquista_September_2025_7.png}
    \caption{Comparison of digital filters - Temperature Sensor 7 (Median)}
    \label{fig:temp_filter_median_7}
\end{figure}

\begin{figure}[H]
    \centering
    \includegraphics[width=0.95\textwidth]{signal_processing/12_Temperature_Analysis_Panel_Complete_View_Raw_vs_EWMA_FilterednVitoria_da_Conquista_September_2025_1.png}
    \caption{Comparison of digital filters - Temperature Sensor 1 (EWMA)}
    \label{fig:temp_filter_ewma_1}
\end{figure}

\begin{figure}[H]
    \centering
    \includegraphics[width=0.95\textwidth]{signal_processing/12_Temperature_Analysis_Panel_Complete_View_Raw_vs_EWMA_FilterednVitoria_da_Conquista_September_2025_2.png}
    \caption{Comparison of digital filters - Temperature Sensor 2 (EWMA)}
    \label{fig:temp_filter_ewma_2}
\end{figure}

\begin{figure}[H]
    \centering
    \includegraphics[width=0.95\textwidth]{signal_processing/12_Temperature_Analysis_Panel_Complete_View_Raw_vs_EWMA_FilterednVitoria_da_Conquista_September_2025_3.png}
    \caption{Comparison of digital filters - Temperature Sensor 3 (EWMA)}
    \label{fig:temp_filter_ewma_3}
\end{figure}

\begin{figure}[H]
    \centering
    \includegraphics[width=0.95\textwidth]{signal_processing/12_Temperature_Analysis_Panel_Complete_View_Raw_vs_EWMA_FilterednVitoria_da_Conquista_September_2025_4.png}
    \caption{Comparison of digital filters - Temperature Sensor 4 (EWMA)}
    \label{fig:temp_filter_ewma_4}
\end{figure}

\begin{figure}[H]
    \centering
    \includegraphics[width=0.95\textwidth]{signal_processing/12_Temperature_Analysis_Panel_Complete_View_Raw_vs_EWMA_FilterednVitoria_da_Conquista_September_2025_5.png}
    \caption{Comparison of digital filters - Temperature Sensor 5 (EWMA)}
    \label{fig:temp_filter_ewma_5}
\end{figure}

\begin{figure}[H]
    \centering
    \includegraphics[width=0.95\textwidth]{signal_processing/12_Temperature_Analysis_Panel_Complete_View_Raw_vs_EWMA_FilterednVitoria_da_Conquista_September_2025_6.png}
    \caption{Comparison of digital filters - Temperature Sensor 6 (EWMA)}
    \label{fig:temp_filter_ewma_6}
\end{figure}

\begin{figure}[H]
    \centering
    \includegraphics[width=0.95\textwidth]{signal_processing/12_Temperature_Analysis_Panel_Complete_View_Raw_vs_EWMA_FilterednVitoria_da_Conquista_September_2025_7.png}
    \caption{Comparison of digital filters - Temperature Sensor 7 (EWMA)}
    \label{fig:temp_filter_ewma_7}
\end{figure}

\subsubsection{Filter Comparison - Humidity}

\begin{figure}[H]
    \centering
    \includegraphics[width=0.95\textwidth]{signal_processing/12_Humidity_Analysis_Panel_Complete_View_Raw_vs_FilterednVitoria_da_Conquista_September_2025_1.png}
    \caption{Comparison of digital filters - Humidity Sensor 1 (Moving Average)}
    \label{fig:humid_filter_ma_1}
\end{figure}

\begin{figure}[H]
    \centering
    \includegraphics[width=0.95\textwidth]{signal_processing/12_Humidity_Analysis_Panel_Complete_View_Raw_vs_FilterednVitoria_da_Conquista_September_2025_2.png}
    \caption{Comparison of digital filters - Humidity Sensor 2 (Moving Average)}
    \label{fig:humid_filter_ma_2}
\end{figure}

\begin{figure}[H]
    \centering
    \includegraphics[width=0.95\textwidth]{signal_processing/12_Humidity_Analysis_Panel_Complete_View_Raw_vs_Median_FilterednVitoria_da_Conquista_September_2025_1.png}
    \caption{Comparison of digital filters - Humidity Sensor 1 (Median)}
    \label{fig:humid_filter_median_1}
\end{figure}

\begin{figure}[H]
    \centering
    \includegraphics[width=0.95\textwidth]{signal_processing/12_Humidity_Analysis_Panel_Complete_View_Raw_vs_Median_FilterednVitoria_da_Conquista_September_2025_2.png}
    \caption{Comparison of digital filters - Humidity Sensor 2 (Median)}
    \label{fig:humid_filter_median_2}
\end{figure}

\begin{figure}[H]
    \centering
    \includegraphics[width=0.95\textwidth]{signal_processing/12_Humidity_Analysis_Panel_Complete_View_Raw_vs_EWMA_FilterednVitoria_da_Conquista_September_2025_1.png}
    \caption{Comparison of digital filters - Humidity Sensor 1 (EWMA)}
    \label{fig:humid_filter_ewma_1}
\end{figure}

\begin{figure}[H]
    \centering
    \includegraphics[width=0.95\textwidth]{signal_processing/12_Humidity_Analysis_Panel_Complete_View_Raw_vs_EWMA_FilterednVitoria_da_Conquista_September_2025_2.png}
    \caption{Comparison of digital filters - Humidity Sensor 2 (EWMA)}
    \label{fig:humid_filter_ewma_2}
\end{figure}

\subsubsection{Filter Comparison - Pressure}

\begin{figure}[H]
    \centering
    \includegraphics[width=0.95\textwidth]{signal_processing/12_Pressure_Analysis_Panel_Complete_View_Raw_vs_FilterednVitoria_da_Conquista_September_2025_1.png}
    \caption{Comparison of digital filters - Pressure Sensor 1 (Moving Average)}
    \label{fig:press_filter_ma_1}
\end{figure}

\begin{figure}[H]
    \centering
    \includegraphics[width=0.95\textwidth]{signal_processing/12_Pressure_Analysis_Panel_Complete_View_Raw_vs_FilterednVitoria_da_Conquista_September_2025_2.png}
    \caption{Comparison of digital filters - Pressure Sensor 2 (Moving Average)}
    \label{fig:press_filter_ma_2}
\end{figure}

\begin{figure}[H]
    \centering
    \includegraphics[width=0.95\textwidth]{signal_processing/12_Pressure_Analysis_Panel_Complete_View_Raw_vs_Median_FilterednVitoria_da_Conquista_September_2025_1.png}
    \caption{Comparison of digital filters - Pressure Sensor 1 (Median)}
    \label{fig:press_filter_median_1}
\end{figure}

\begin{figure}[H]
    \centering
    \includegraphics[width=0.95\textwidth]{signal_processing/12_Pressure_Analysis_Panel_Complete_View_Raw_vs_Median_FilterednVitoria_da_Conquista_September_2025_2.png}
    \caption{Comparison of digital filters - Pressure Sensor 2 (Median)}
    \label{fig:press_filter_median_2}
\end{figure}

\begin{figure}[H]
    \centering
    \includegraphics[width=0.95\textwidth]{signal_processing/12_Pressure_Analysis_Panel_Complete_View_Raw_vs_EWMA_FilterednVitoria_da_Conquista_September_2025_1.png}
    \caption{Comparison of digital filters - Pressure Sensor 1 (EWMA)}
    \label{fig:press_filter_ewma_1}
\end{figure}

\begin{figure}[H]
    \centering
    \includegraphics[width=0.95\textwidth]{signal_processing/12_Pressure_Analysis_Panel_Complete_View_Raw_vs_EWMA_FilterednVitoria_da_Conquista_September_2025_2.png}
    \caption{Comparison of digital filters - Pressure Sensor 2 (EWMA)}
    \label{fig:press_filter_ewma_2}
\end{figure}

\textbf{Digital filtering results:}
\begin{itemize}
    \item \textbf{Median Filter:} Best noise suppression without phase lag
    \item \textbf{EWMA:} Smooth transitions, preserves trends
    \item \textbf{Moving Average:} Trade-off between smoothing and responsiveness
    \item Complete panels with raw, filtered signal, zoom, and frequency spectrum
\end{itemize}

\subsection{Notebook 13: Frequency Analysis (FFT)}

\textbf{Note:} FFT analyses were integrated into the digital filter panels (Notebook 12), showing the power spectrum of each signal before and after filtering.

\textbf{Spectral analysis results:}
\begin{itemize}
    \item \textbf{Primary peak:} 23-hour periodicity (diurnal cycle)
    \item Clear spectral signature of daily heating/cooling cycle
    \item Validation of the adequacy of the 30-second sampling interval
\end{itemize}

\section{Public Dataset}

The complete dataset was made publicly available on Kaggle:

\begin{itemize}
    \item \textbf{Title:} ``Vitória da Conquista Weather Data - September 2025''
    \item \textbf{URL:} \url{https://www.kaggle.com/datasets/jonassouza872/vitoria-da-conquista-weather-data-september}
    \item \textbf{Records:} 82,430 measurements
    \item \textbf{Format:} CSV with 12 environmental variables
    \item \textbf{License:} Free, public access for download
\end{itemize}

\section{Repository Structure}

\begin{verbatim}
PolySense-Station-/
|-- main.py                          # 226 lines MicroPython
|-- requirements.txt                 # Python dependencies
|-- lib/                             # MicroPython drivers (9 files)
|-- notebooks/                       # 13 analysis notebooks
|   |-- 01_exploratory_analysis.ipynb
|   |-- 02_correlation_analysis.ipynb
|   |-- 03_missing_data.ipynb
|   |-- 04_sensor_validation.ipynb
|   |-- 05_temporal_analysis.ipynb
|   |-- 06_time_series_decomposition.ipynb
|   |-- 07_anomaly_detection.ipynb
|   |-- 08_decision_tree_regression.ipynb
|   |-- 09_gmm_clustering.ipynb
|   |-- 10_kmeans_clustering.ipynb
|   |-- 11_lstm_prediction.ipynb
|   |-- 12_digital_filters.ipynb
|   |-- 13_fft_analysis.ipynb
|-- data/raw/                        # 4 CSV files (4.8 MB total)
|   |-- climate_clusters_gmm.csv     # 82,430 records
|   |-- validation_data_cleaned_BRT.csv
|   |-- validation_and_Measured_Data_cleaned_BRT_.csv
|   |-- inmet_weather_station_data_sep_2025_utc.csv
|-- images/                          # 111 generated visualizations
|   |-- data_analysis/               # EDA, correlation, validation
|   |-- machine_learning/            # Clustering, anomalies, regression
|   |-- signal_processing/           # Filter comparisons, FFT
|-- PCB/                             # Custom PCB design
|-- Schematic/                       # Hardware schematics
|-- README.md                        # Complete documentation
\end{verbatim}

\section{Conclusions}

\subsection{Main Achievements}

\begin{enumerate}
    \item \textbf{Functional collection system:} 100\% success over 30 days (82,430 records at 30s interval)

    \item \textbf{Effective sensor redundancy:} 7 temperature sensors, 2 humidity, 2 pressure with >99.9\% completeness

    \item \textbf{Validation against reference:} Mean bias < 2°C compared to official INMET station

    \item \textbf{Exceptional correlation:} r > 0.98 between redundant sensors

    \item \textbf{Anomaly detection:} 3.2\% outliers successfully isolated with Isolation Forest

    \item \textbf{Accurate prediction:} MAE < 1°C for 1-hour ahead temperature prediction with LSTM

    \item \textbf{Climate clustering:} 4 distinct regimes identified and validated (GMM and KMeans)

    \item \textbf{Spectral analysis:} Clear 24h periodicity confirmed via FFT

    \item \textbf{Public dataset:} 82,430 measurements available on Kaggle for the scientific community
\end{enumerate}

\subsection{Study Limitations}

\begin{itemize}
    \item Data collection limited to 30 days (September 2025)
    \item Lack of direct solar radiation measurement
    \item Temporal granularity of 30 seconds (does not capture very fast events)
    \item External factors not considered (detailed cloud cover, local wind)
    \item Validation limited to one season
\end{itemize}

\subsection{Future Work}

\begin{enumerate}
    \item Expand collection to 12 months (complete seasonal analysis)
    \item Add solar radiation sensor (pyranometer)
    \item Implement anemometer for wind speed and direction
    \item Develop PCB version 2 with identified improvements
    \item Integrate data transmission via LoRaWAN
    \item Implement edge computing with real-time LSTM predictions
    \item Comparative analysis between multiple stations at different altitudes
    \item Develop REST API for real-time data access
\end{enumerate}

\section{References}

\begin{itemize}
    \item \textbf{Microcontroller:} Raspberry Pi Pico (RP2040), MicroPython Documentation
    \item \textbf{Reference station:} INMET - National Institute of Meteorology
    \item \textbf{Dataset:} \url{https://www.kaggle.com/datasets/jonassouza872/vitoria-da-conquista-weather-data-september}
    \item \textbf{Period:} 08/31/2025 - 09/30/2025 (82,430 records)
    \item \textbf{Python libraries:} Pandas, NumPy, Matplotlib, Seaborn, Scikit-learn, TensorFlow, SciPy
    \item \textbf{ML techniques:} Isolation Forest, Decision Tree, KMeans, GMM, LSTM
    \item \textbf{Signal processing:} FFT, Digital filters (Moving average, Median, EWMA)
\end{itemize}

\end{document}
