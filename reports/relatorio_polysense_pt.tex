\documentclass[12pt,a4paper]{article}
\usepackage[utf8]{inputenc}
\usepackage[brazilian]{babel}
\usepackage{graphicx}
\usepackage{amsmath}
\usepackage{booktabs}
\usepackage{float}
\usepackage{geometry}
\usepackage{caption}
\usepackage{subcaption}
\usepackage{longtable}
\usepackage{array}
\usepackage{url}

\geometry{margin=2.5cm}

% Configurar caminho das imagens
\graphicspath{{../images/},{../PCB/Photos/},{../Schematic/}}

\begin{document}

% ============================================
% CUSTOM COVER PAGE
% ============================================
\begin{titlepage}
    \centering
    \vspace*{2cm}

    {\large Engenharia Elétrica}\\[3cm]

    % Title
    {\huge \textbf{PolySense Station}}\\[0.5cm]
    {\Large \textbf{Sistema de Monitoramento Ambiental}}\\[3cm]

    % Subtitle
    {\large Coleta de dados multi-sensor, análise temporal e aprendizado de máquina}\\[0.3cm]
    {\large aplicados a monitoramento climático}\\[4cm]

    \vfill

    % Author information
    \begin{flushleft}
        \large
        \textbf{Autor:} Jonas Souza\\
        \textbf{Formação:} Engenheiro Elétrica\\
        \textbf{Data:} \today
    \end{flushleft}

    \vspace{1cm}

\end{titlepage}

\newpage

\tableofcontents
\newpage

\section{Sumário Executivo}

Este relatório apresenta a análise completa do projeto \textbf{PolySense Station}, um sistema autônomo de aquisição e análise de dados ambientais desenvolvido do zero. O sistema combina hardware customizado com 7 sensores integrados, microcontrolador Raspberry Pi Pico e um pipeline completo de análise de dados com 13 notebooks especializados.

\subsection{Principais Realizações}

\begin{itemize}
    \item \textbf{Coleta de dados real:} 82.430 medições ao longo de 30 dias (setembro de 2025)
    \item \textbf{Resolução temporal:} Intervalo de amostragem de 30 segundos
    \item \textbf{Redundância de sensores:} 7 sensores de temperatura, 2 de umidade, 2 de pressão
    \item \textbf{Validação contra estação INMET:} Instituto Nacional de Meteorologia
    \item \textbf{Acurácia do modelo LSTM:} MAE < 1°C para previsão de temperatura 1 hora à frente
    \item \textbf{Clustering climático:} 4 regimes distintos identificados via GMM
    \item \textbf{Dataset público:} Disponibilizado no Kaggle com 82.430 registros
\end{itemize}

\section{Introdução}

\subsection{Contexto}

O monitoramento ambiental é fundamental para compreender padrões climáticos locais, validar modelos preditivos e desenvolver sistemas de automação sensíveis ao clima. Este projeto implementa uma estação meteorológica completa com coleta automática, armazenamento em cartão SD e análise avançada de dados.

\subsection{Objetivos}

\begin{enumerate}
    \item Desenvolver hardware de aquisição de dados com redundância de sensores
    \item Coletar dados ambientais (30 segundos de intervalo)
    \item Validar medições contra estação meteorológica oficial (INMET)
    \item Realizar análise exploratória completa dos dados coletados
    \item Aplicar técnicas de machine learning para identificação de padrões
    \item Desenvolver modelos preditivos de temperatura com redes neurais LSTM
    \item Aplicar processamento digital de sinais para redução de ruído
\end{enumerate}

\subsection{Localização e Período}

\begin{itemize}
    \item \textbf{Local:} Vitória da Conquista, Bahia, Brasil
    \item \textbf{Altitude:} 923 metros
    \item \textbf{Período de coleta:} 31 de agosto a 30 de setembro de 2025
    \item \textbf{Total de registros:} 82.430 medições
    \item \textbf{Tamanho do dataset:} 4.8 MB (CSV bruto)
\end{itemize}

\section{Metodologia de Coleta de Dados}

\subsection{Arquitetura de Hardware}

O sistema foi desenvolvido utilizando o microcontrolador \textbf{Raspberry Pi Pico} (RP2040) programado em MicroPython, com as seguintes interfaces:

\begin{itemize}
    \item \textbf{Dual I2C:} Dois barramentos I2C para minimizar interferências elétricas
    \item \textbf{SPI:} Interface para gravação em cartão SD
    \item \textbf{OneWire:} Protocolo para sensor de temperatura DS18B20
    \item \textbf{Analógico:} Conversor ADC para termistor NTC
    \item \textbf{Display OLED:} Tela 128x64 para monitoramento em tempo real
    \item \textbf{RTC:} Relógio de tempo real para timestamps precisos
\end{itemize}

\textbf{Design de Hardware:} O sistema inclui uma PCB customizada para integração dos sensores e um esquemático detalhado para replicação:

\begin{figure}[H]
    \centering
    \includegraphics[width=0.95\textwidth]{PCB_front.jpg}
    \caption{Fotografia da PCB front (Printed Circuit Board) do PolySense Station}
    \label{fig:pcb_front}
\end{figure}

\begin{figure}[H]
    \centering
    \includegraphics[width=0.95\textwidth]{Schematic_Protoboard.png}
    \caption{Esquemático do circuito protoboard do PolySense Station}
    \label{fig:schematic_protoboard}
\end{figure}

\begin{figure}[H]
    \centering
    \includegraphics[width=0.95\textwidth]{Schematic_Sensor.png}
    \caption{Esquemático detalhado dos sensores integrados}
    \label{fig:schematic_sensor}
\end{figure}

\subsection{Suite de Sensores Implementada}

\begin{table}[H]
\centering
\caption{Sensores integrados no PolySense Station}
\scriptsize
\begin{tabular}{@{}llll@{}}
\toprule
\textbf{Sensor} & \textbf{Tipo} & \textbf{Medições} & \textbf{Protocolo} \\ \midrule
MPU6050 & Giroscópio/Acelerômetro & Temperatura & I2C \\
AHT20 & Ambiental & Temperatura, Umidade & I2C \\
BMP280 & Barométrico & Temperatura, Pressão & I2C \\
BMP180 & Barométrico & Temperatura, Pressão & I2C \\
DS18B20 & OneWire Digital & Temperatura & OneWire \\
NTC (KY-028) & Termistor Analógico & Temperatura & ADC \\
DHT11 & Ambiental & Temperatura, Umidade & Digital \\ \bottomrule
\end{tabular}
\label{tab:sensores}
\end{table}

\textbf{Justificativa da redundância:}
\begin{itemize}
    \item Validação cruzada entre múltiplos sensores
    \item Detecção de falhas instrumentais
    \item Referências (DS18B20, BMP280, AHT20)
\end{itemize}

\subsection{Firmware de Aquisição}

O firmware principal (\texttt{main.py}, 226 linhas) implementa:

\begin{itemize}
    \item Leitura síncrona de todos os sensores a cada 30 segundos
    \item Gravação em cartão SD com timestamp RTC
    \item Display OLED rotativo com 3 telas de informação
    \item LED de feedback (pisca a cada gravação bem-sucedida)
    \item Sistema de ejeção segura via botão
    \item Tratamento de erros e status de gravação
\end{itemize}

\textbf{Formato de armazenamento CSV:}
\begin{verbatim}
Timestamp, Temp_MPU6050_C, Temp_AHT20_C, Umid_AHT20_pct,
Temp_BMP280_C, Press_BMP280_hPa, Temp_BMP180_C,
Press_BMP180_hPa, Temp_DS18B20_C, Temp_NTC_C,
Temp_DHT11_C, Umid_DHT11_pct
\end{verbatim}

\begin{figure}[H]
    \centering
    \includegraphics[width=0.95\textwidth]{Serial_output.png}
    \caption{Saída serial do firmware durante operação do PolySense Station}
    \label{fig:serial_output}
\end{figure}

\subsection{Validação Contra Estação INMET}

Os dados foram validados contra a estação meteorológica oficial do INMET:
\begin{itemize}
    \item \textbf{Dataset de validação:} 719 registros da estação INMET
    \item \textbf{Variáveis INMET:} Temperatura, umidade, pressão, vento, radiação, precipitação
    \item \textbf{Dataset combinado:} 331 registros horários agregados
\end{itemize}

\section{Módulo 1: Análise Exploratória de Dados (Notebooks 01-05)}

\subsection{Notebook 01: Análise Exploratória Inicial}

\subsubsection{Distribuições de Temperatura}

Análise das distribuições de temperatura revelou que todos os 7 sensores apresentam padrões consistentes:

\begin{figure}[H]
    \centering
    \includegraphics[width=0.95\textwidth]{data_analysis/01_Temperature_Data_Histogram_in_September_Vitória_da_Conquista.png}
    \caption{Histograma de distribuições de temperatura (7 sensores)}
    \label{fig:temp_hist}
\end{figure}

\textbf{Observações:}
\begin{itemize}
    \item Distribuições aproximadamente normais com picos entre 20-22°C
    \item Sensores(BMP280, AHT20, DS18B20) com curvas suaves
    \item DHT11 exibe artefatos de quantização (degraus discretos)
    \item Termistor NTC mostra maior dispersão (sensibilidade à radiação solar)
\end{itemize}

\begin{figure}[H]
    \centering
    \includegraphics[width=0.95\textwidth]{data_analysis/01_Comparison_of_Temperature_Sensor_Distributions_September_Vitória_da_Conquista.png}
    \caption{Boxplot comparativo - Distribuição de temperatura entre sensores}
    \label{fig:temp_boxplot}
\end{figure}

\textbf{Principais achados:}
\begin{itemize}
    \item Excelente concordância entre sensores (medianas muito próximas)
    \item Outliers superiores identificam picos de temperatura diurnos
    \item Intervalo interquartil consistente entre sensore.
    \item DHT11 apresenta maior variabilidade (menor resolução)
\end{itemize}

\subsubsection{Distribuições de Umidade}

\begin{figure}[H]
    \centering
    \includegraphics[width=0.95\textwidth]{data_analysis/01_Humidity_Data_Histogram_in_September_Vitória_da_Conquista.png}
    \caption{Histograma de umidade relativa (AHT20 e DHT11)}
    \label{fig:humid_hist}
\end{figure}

\textbf{Padrão bimodal identificado:}
\begin{itemize}
    \item Primeiro pico: 50-55\% (período diurno)
    \item Segundo pico: 75-80\% (período noturno com formação de neblina)
    \item Consistente com clima de setembro no planalto (altitude 923m)
\end{itemize}

\begin{figure}[H]
    \centering
    \includegraphics[width=0.95\textwidth]{data_analysis/01_Comparison_of_Humidity_Sensor_Distributions_September_Vitória_da_Conquista.png}
    \caption{Boxplot comparativo de umidade (AHT20 vs DHT11)}
    \label{fig:humid_boxplot}
\end{figure}

\subsubsection{Distribuições de Pressão Barométrica}

\begin{figure}[H]
    \centering
    \includegraphics[width=0.95\textwidth]{data_analysis/01_Pressure_Data_Histogram_in_September_Vitória_da_Conquista.png}
    \caption{Histograma de pressão barométrica (BMP280 e BMP180)}
    \label{fig:press_hist}
\end{figure}

\textbf{Características observadas:}
\begin{itemize}
    \item Faixa: 914-924 hPa (média ~920 hPa)
    \item Distribuição consistente com altitude de 923m
    \item Excelente concordância entre BMP280 e BMP180
\end{itemize}

\begin{figure}[H]
    \centering
    \includegraphics[width=0.95\textwidth]{data_analysis/01_Comparison_of_Pressure_Sensor_Distributions_September_Vitória_da_Conquista.png}
    \caption{Boxplot comparativo de pressão (BMP280 vs BMP180)}
    \label{fig:press_boxplot}
\end{figure}

\subsection{Notebook 02: Análise de Correlações}

\subsubsection{Matriz de Correlação Global}

\begin{figure}[H]
    \centering
    \includegraphics[width=0.95\textwidth]{data_analysis/02_Correlation_Matrix_All_Environmental_SensorsnVitória_da_Conquista_September_2025.png}
    \caption{Matriz de correlação entre todas as 11 variáveis medidas}
    \label{fig:corr_matrix}
\end{figure}

\textbf{Correlações identificadas:}
\begin{itemize}
    \item \textbf{Sensores de temperatura (entre si):} r > 0.99
    \item \textbf{Sensores de umidade (AHT20 vs DHT11):} r = 0.985
    \item \textbf{Sensores de pressão (BMP280 vs BMP180):} r = 0.984
    \item \textbf{Temperatura vs Umidade:} r = -0.868 (forte inversa)
    \item \textbf{Temperatura vs Pressão:} r = -0.559 (moderada inversa)
    \item \textbf{Umidade vs Pressão:} r = 0.463 (fraca positiva)
\end{itemize}

\subsubsection{Relações Bivariadas com KDE}

\begin{figure}[H]
    \centering
    \includegraphics[width=0.95\textwidth]{data_analysis/02_Relationship_between_Temperature_BMP280_and_Humidity_AHT20nVitória_da_Conquista_September_2025.png}
    \caption{Relação Temperatura vs Umidade com densidade de probabilidade (KDE)}
    \label{fig:temp_humid_kde}
\end{figure}

\textbf{Interpretação física:}
\begin{itemize}
    \item Correlação negativa forte (r = -0.868)
    \item Aquecimento diurno reduz umidade relativa
    \item Resfriamento noturno aumenta umidade (formação de neblina)
\end{itemize}

\begin{figure}[H]
    \centering
    \includegraphics[width=0.95\textwidth]{data_analysis/02_Relationship_between_Temperature_BMP280_and_Pressure_BMP180nVitória_da_Conquista_September_2025.png}
    \caption{Relação Temperatura vs Pressão Barométrica}
    \label{fig:temp_press_kde}
\end{figure}

\begin{figure}[H]
    \centering
    \includegraphics[width=0.95\textwidth]{data_analysis/02_Relationship_between_Humidity_AHT20_and_Pressure_BMP180nVitória_da_Conquista_September_2025.png}
    \caption{Relação Umidade vs Pressão Barométrica}
    \label{fig:humid_press_kde}
\end{figure}

\subsection{Notebook 03: Análise de Dados Faltantes}

\begin{figure}[H]
    \centering
    \includegraphics[width=0.9\textwidth]{data_analysis/03_Data_Completeness_per_Temperature_Sensor_Vitória_da_Conquista_September_2025.png}
    \caption{Completude de dados - Sensores de temperatura}
    \label{fig:completeness_temp}
\end{figure}

\begin{figure}[H]
    \centering
    \includegraphics[width=0.9\textwidth]{data_analysis/03_Data_Completeness_per_Humidity_Sensor_Vitória_da_Conquista_September_2025.png}
    \caption{Completude de dados - Sensores de umidade}
    \label{fig:completeness_humid}
\end{figure}

\begin{figure}[H]
    \centering
    \includegraphics[width=0.9\textwidth]{data_analysis/03_Data_Completeness_per_Pressure_Sensor_Vitória_da_Conquista_September_2025.png}
    \caption{Completude de dados - Sensores de pressão}
    \label{fig:completeness_press}
\end{figure}

\textbf{Resultados de qualidade:}
\begin{itemize}
    \item DS18B20: Maior completude (>99.99\%)
    \item DHT11: Pequenas lacunas (~99.98\%)
    \item Todos os sensores: >99.9\% de dados disponíveis
    \item Alta confiabilidade do sistema de aquisição
\end{itemize}

\subsection{Notebook 04: Validação de Sensores}

\subsubsection{Análise de Bland-Altman - Temperatura}

\begin{figure}[H]
    \centering
    \includegraphics[width=0.95\textwidth]{data_analysis/04_BLAND_ALTMAN_ANALYSIS_Temperature_Sensors_Pairwise_Comparison.png}
    \caption{Bland-Altman: Comparação entre pares de sensores de temperatura}
    \label{fig:bland_altman_temp}
\end{figure}

\textbf{Validação inter-sensor:}
\begin{itemize}
    \item Bias médio < 0.5°C entre sensores.
    \item Limites de concordância dentro de ±2°C
    \item Excelente concordância entre BMP280, AHT20 e DS18B20
\end{itemize}

\begin{figure}[H]
    \centering
    \includegraphics[width=0.95\textwidth]{data_analysis/04_BLAND_ALTMAN_ANALYSIS_Sensors_vs_Reference_Temp_Ins_C.png}
    \caption{Bland-Altman: Sensores vs Referência INMET}
    \label{fig:bland_altman_inmet}
\end{figure}

\textbf{Validação contra INMET:}
\begin{itemize}
    \item Bias médio < 2°C em relação à estação oficial
    \item Termistor NTC apresenta offset sistemático (radiação solar)
    \item Sensores digitais validam confiavelmente
\end{itemize}

\subsubsection{Análise de Bland-Altman - Umidade e Pressão}

\begin{figure}[H]
    \centering
    \includegraphics[width=0.95\textwidth]{data_analysis/04_BLAND_ALTMAN_ANALYSIS_Humidity_Sensors_Pairwise_Comparison.png}
    \caption{Bland-Altman: Comparação de sensores de umidade (AHT20 vs DHT11)}
    \label{fig:bland_altman_humid}
\end{figure}

\begin{figure}[H]
    \centering
    \includegraphics[width=0.95\textwidth]{data_analysis/04_BLAND_ALTMAN_ANALYSIS_Pressure_Sensors_BMP280_vs_BMP180.png}
    \caption{Bland-Altman: Comparação de sensores de pressão}
    \label{fig:bland_altman_press}
\end{figure}

\subsection{Notebook 05: Análise Temporal}

\subsubsection{Séries Temporais Completas}

\begin{figure}[H]
    \centering
    \includegraphics[width=0.95\textwidth]{data_analysis/05_Temperature_Sensors_Comparison_September_2025_Vitória_da_Conquista.png}
    \caption{Série temporal completa de temperatura (setembro 2025)}
    \label{fig:temp_ts}
\end{figure}

\textbf{Padrões identificados:}
\begin{itemize}
    \item Ciclo diurno claro com amplitude de 10-12°C
    \item Pico de temperatura: 14:00-16:00 hora local
    \item Mínimo de temperatura: 06:00-07:00 (antes do amanhecer)
    \item Variações semanais associadas a sistemas sinóticos
\end{itemize}

\begin{figure}[H]
    \centering
    \includegraphics[width=0.95\textwidth]{data_analysis/05_Humidity_Sensors_Comparison_September_2025_Vitória_da_Conquista.png}
    \caption{Série temporal de umidade (setembro 2025)}
    \label{fig:humid_ts}
\end{figure}

\begin{figure}[H]
    \centering
    \includegraphics[width=0.95\textwidth]{data_analysis/05_Pressure_Sensors_Comparison_September_2025_Vitória_da_Conquista.png}
    \caption{Série temporal de pressão barométrica (setembro 2025)}
    \label{fig:press_ts}
\end{figure}

\textbf{Observações de pressão:}
\begin{itemize}
    \item Oscilações de 10-12 hPa ao longo do mês
    \item Variações acompanham passagens de frentes
    \item Pressão média de 920 hPa (consistente com altitude)
\end{itemize}

\subsection{Notebook 06: Decomposição de Séries Temporais}

\begin{figure}[H]
    \centering
    \includegraphics[width=0.95\textwidth]{data_analysis/06_Time_Series_Decomposition_plot_01.png}
    \caption{Decomposição de série temporal - Temperatura}
    \label{fig:decomp_temp}
\end{figure}

\textbf{Componentes identificados:}
\begin{itemize}
    \item \textbf{Tendência:} Leve aquecimento até meados de setembro, depois resfriamento
    \item \textbf{Sazonal:} Forte periodicidade de 24 horas
    \item \textbf{Resíduos:} Limpos com poucas anomalias
\end{itemize}

\begin{figure}[H]
    \centering
    \includegraphics[width=0.95\textwidth]{data_analysis/06_Time_Series_Decomposition_plot_03.png}
    \caption{Decomposição de série temporal - Umidade}
    \label{fig:decomp_humid}
\end{figure}

\begin{figure}[H]
    \centering
    \includegraphics[width=0.95\textwidth]{data_analysis/06_Time_Series_Decomposition_plot_05.png}
    \caption{Decomposição de série temporal - Pressão}
    \label{fig:decomp_press}
\end{figure}

\section{Módulo 2: Machine Learning (Notebooks 07-11)}

\subsection{Notebook 07: Detecção de Anomalias}

\subsubsection{Isolation Forest}

\begin{figure}[H]
    \centering
    \includegraphics[width=0.95\textwidth]{machine_learning/07_Isolation_Forest_Anomaly_Detection_Across_All_Sensors.png}
    \caption{Detecção de anomalias via Isolation Forest (todos os sensores)}
    \label{fig:isolation_forest}
\end{figure}

\textbf{Resultados:}
\begin{itemize}
    \item \textbf{3.2\% das medições} classificadas como anomalias
    \item Anomalias concentradas em janelas temporais específicas
    \item Isolamento bem-sucedido de falhas instrumentais
\end{itemize}

\begin{figure}[H]
    \centering
    \includegraphics[width=0.95\textwidth]{machine_learning/07_Heatmap_Sensor_Deviations_in_Detected_AnomaliesnRed_High_deviation_Green_Normal_showing_50_random_anomalies.png}
    \caption{Heatmap de desvios de sensores em anomalias detectadas}
    \label{fig:anomaly_heatmap}
\end{figure}

\textbf{Análise de desvios:}
\begin{itemize}
    \item Vermelho: Alta divergência em relação à média
    \item Verde: Comportamento normal
    \item Identificação de quais sensores desviaram durante anomalias
\end{itemize}

\subsubsection{Detecção de Valores Congelados}

\begin{figure}[H]
    \centering
    \includegraphics[width=0.95\textwidth]{machine_learning/07_Stuck_Values_Detection_Δ_001C.png}
    \caption{Detecção de valores congelados (sensor freezes, $\Delta < 0.01$°C)}
    \label{fig:stuck_values}
\end{figure}

\textbf{Critério de detecção:}
\begin{itemize}
    \item Identificação de medições onde $\Delta < 0.01$°C por >10 minutos
    \item Detecção de falhas temporárias de sensores
    \item Validação da necessidade de redundância multi-sensor
\end{itemize}

\subsection{Notebook 08: Regressão com Árvore de Decisão}

\begin{figure}[H]
    \centering
    \includegraphics[width=0.95\textwidth]{machine_learning/08_Decision_Tree_Temperature_Prediction_All_Sensors_Average_September_2025_Analysis_in_Vitória_da_Conquista.png}
    \caption{Estrutura da árvore de decisão para previsão de temperatura}
    \label{fig:decision_tree}
\end{figure}

\begin{figure}[H]
    \centering
    \includegraphics[width=0.9\textwidth]{machine_learning/08_Which_Variable_Most_Influences_the_Analysis_Vitória_da_Conquista_September_2025.png}
    \caption{Importância de features para previsão de temperatura}
    \label{fig:feature_importance}
\end{figure}

\textbf{Features mais importantes:}
\begin{itemize}
    \item AHT20 e BMP280 entre os principais preditores
    \item Umidade tem forte poder preditivo para temperatura
    \item Modelo apresenta regras interpretáveis
\end{itemize}

\subsection{Notebook 09: Clustering com Gaussian Mixture Model (GMM)}

\begin{figure}[H]
    \centering
    \includegraphics[width=0.95\textwidth]{machine_learning/09_Gaussian_Mixture_Clustering_plot_01.png}
    \caption{Clustering GMM - Visualização 2D (K=4 regimes climáticos)}
    \label{fig:gmm_2d}
\end{figure}

\begin{figure}[H]
    \centering
    \includegraphics[width=0.95\textwidth]{machine_learning/09_3D_Climate_Clustering_GMM_K4nVitória_da_Conquista_September_2025.png}
    \caption{Clustering GMM - Visualização 3D (Temperatura, Umidade, Pressão)}
    \label{fig:gmm_3d}
\end{figure}

\textbf{4 regimes climáticos identificados:}
\begin{enumerate}
    \item \textbf{Cluster 1 (Azul):} Frio-Úmido (noturno com neblina)
    \item \textbf{Cluster 2 (Laranja):} Quente-Seco (tarde com máximas)
    \item \textbf{Cluster 3 (Verde):} Transição matinal (aquecimento progressivo)
    \item \textbf{Cluster 4 (Vermelho):} Transição vespertina (resfriamento)
\end{enumerate}

\subsection{Notebook 10: Clustering com KMeans}

\begin{figure}[H]
    \centering
    \includegraphics[width=0.95\textwidth]{machine_learning/10_KMeans_Clustering_plot_01.png}
    \caption{KMeans clustering - Visualização 2D (K=4)}
    \label{fig:kmeans_2d}
\end{figure}

\begin{figure}[H]
    \centering
    \includegraphics[width=0.95\textwidth]{machine_learning/10_3D_Climate_Clustering_K_Means_K4nVitória_da_Conquista_September_2025.png}
    \caption{KMeans clustering - Visualização 3D}
    \label{fig:kmeans_3d}
\end{figure}

\textbf{Consistência entre métodos:}
\begin{itemize}
    \item KMeans valida os 4 estados climáticos identificados pelo GMM
    \item Clusters bem separados no espaço tridimensional
    \item Confirmação de padrões climáticos ao longo de setembro
\end{itemize}

\subsection{Notebook 11: Previsão com Redes Neurais LSTM}

\begin{figure}[H]
    \centering
    \includegraphics[width=0.9\textwidth]{machine_learning/11_Error_Evolution_During_Training.png}
    \caption{Evolução do erro durante treinamento do modelo LSTM}
    \label{fig:lstm_loss}
\end{figure}

\textbf{Características do modelo:}
\begin{itemize}
    \item Arquitetura LSTM com janela temporal de 30 minutos
    \item Convergência observada após ~50 épocas
    \item Redução consistente de erro de treinamento e validação
\end{itemize}

\begin{figure}[H]
    \centering
    \includegraphics[width=0.95\textwidth]{machine_learning/11_Real_Temperature_vs_Model_Prediction_Vitória_da_Conquista_in_the_Last_Week_of_September_2025.png}
    \caption{Temperatura Real vs Previsões LSTM (última semana de setembro 2025)}
    \label{fig:lstm_predictions}
\end{figure}

\textbf{Desempenho do modelo:}
\begin{itemize}
    \item \textbf{MAE < 1°C} para previsão 1 hora à frente
    \item Modelo captura padrões diurnos com precisão
    \item Previsões se ajustam às temperaturas reais no período de validação
    \item Dependências temporais capturadas efetivamente
\end{itemize}

\section{Módulo 3: Processamento Digital de Sinais (Notebooks 12-13)}

\subsection{Notebook 12: Filtros Digitais}

\subsubsection{Comparação de Filtros - Temperatura}

\begin{figure}[H]
    \centering
    \includegraphics[width=0.95\textwidth]{signal_processing/12_Temperature_Analysis_Panel_Complete_View_Raw_vs_FilterednVitoria_da_Conquista_September_2025_1.png}
    \caption{Comparação de filtros digitais - Temperatura Sensor 1 (Média Móvel)}
    \label{fig:temp_filter_ma_1}
\end{figure}

\begin{figure}[H]
    \centering
    \includegraphics[width=0.95\textwidth]{signal_processing/12_Temperature_Analysis_Panel_Complete_View_Raw_vs_FilterednVitoria_da_Conquista_September_2025_2.png}
    \caption{Comparação de filtros digitais - Temperatura Sensor 2 (Média Móvel)}
    \label{fig:temp_filter_ma_2}
\end{figure}

\begin{figure}[H]
    \centering
    \includegraphics[width=0.95\textwidth]{signal_processing/12_Temperature_Analysis_Panel_Complete_View_Raw_vs_FilterednVitoria_da_Conquista_September_2025_3.png}
    \caption{Comparação de filtros digitais - Temperatura Sensor 3 (Média Móvel)}
    \label{fig:temp_filter_ma_3}
\end{figure}

\begin{figure}[H]
    \centering
    \includegraphics[width=0.95\textwidth]{signal_processing/12_Temperature_Analysis_Panel_Complete_View_Raw_vs_FilterednVitoria_da_Conquista_September_2025_4.png}
    \caption{Comparação de filtros digitais - Temperatura Sensor 4 (Média Móvel)}
    \label{fig:temp_filter_ma_4}
\end{figure}

\begin{figure}[H]
    \centering
    \includegraphics[width=0.95\textwidth]{signal_processing/12_Temperature_Analysis_Panel_Complete_View_Raw_vs_FilterednVitoria_da_Conquista_September_2025_5.png}
    \caption{Comparação de filtros digitais - Temperatura Sensor 5 (Média Móvel)}
    \label{fig:temp_filter_ma_5}
\end{figure}

\begin{figure}[H]
    \centering
    \includegraphics[width=0.95\textwidth]{signal_processing/12_Temperature_Analysis_Panel_Complete_View_Raw_vs_FilterednVitoria_da_Conquista_September_2025_6.png}
    \caption{Comparação de filtros digitais - Temperatura Sensor 6 (Média Móvel)}
    \label{fig:temp_filter_ma_6}
\end{figure}

\begin{figure}[H]
    \centering
    \includegraphics[width=0.95\textwidth]{signal_processing/12_Temperature_Analysis_Panel_Complete_View_Raw_vs_FilterednVitoria_da_Conquista_September_2025_7.png}
    \caption{Comparação de filtros digitais - Temperatura Sensor 7 (Média Móvel)}
    \label{fig:temp_filter_ma_7}
\end{figure}

\begin{figure}[H]
    \centering
    \includegraphics[width=0.95\textwidth]{signal_processing/12_Temperature_Analysis_Panel_Complete_View_Raw_vs_Median_FilterednVitoria_da_Conquista_September_2025_1.png}
    \caption{Comparação de filtros digitais - Temperatura Sensor 1 (Mediana)}
    \label{fig:temp_filter_median_1}
\end{figure}

\begin{figure}[H]
    \centering
    \includegraphics[width=0.95\textwidth]{signal_processing/12_Temperature_Analysis_Panel_Complete_View_Raw_vs_Median_FilterednVitoria_da_Conquista_September_2025_2.png}
    \caption{Comparação de filtros digitais - Temperatura Sensor 2 (Mediana)}
    \label{fig:temp_filter_median_2}
\end{figure}

\begin{figure}[H]
    \centering
    \includegraphics[width=0.95\textwidth]{signal_processing/12_Temperature_Analysis_Panel_Complete_View_Raw_vs_Median_FilterednVitoria_da_Conquista_September_2025_3.png}
    \caption{Comparação de filtros digitais - Temperatura Sensor 3 (Mediana)}
    \label{fig:temp_filter_median_3}
\end{figure}

\begin{figure}[H]
    \centering
    \includegraphics[width=0.95\textwidth]{signal_processing/12_Temperature_Analysis_Panel_Complete_View_Raw_vs_Median_FilterednVitoria_da_Conquista_September_2025_4.png}
    \caption{Comparação de filtros digitais - Temperatura Sensor 4 (Mediana)}
    \label{fig:temp_filter_median_4}
\end{figure}

\begin{figure}[H]
    \centering
    \includegraphics[width=0.95\textwidth]{signal_processing/12_Temperature_Analysis_Panel_Complete_View_Raw_vs_Median_FilterednVitoria_da_Conquista_September_2025_5.png}
    \caption{Comparação de filtros digitais - Temperatura Sensor 5 (Mediana)}
    \label{fig:temp_filter_median_5}
\end{figure}

\begin{figure}[H]
    \centering
    \includegraphics[width=0.95\textwidth]{signal_processing/12_Temperature_Analysis_Panel_Complete_View_Raw_vs_Median_FilterednVitoria_da_Conquista_September_2025_6.png}
    \caption{Comparação de filtros digitais - Temperatura Sensor 6 (Mediana)}
    \label{fig:temp_filter_median_6}
\end{figure}

\begin{figure}[H]
    \centering
    \includegraphics[width=0.95\textwidth]{signal_processing/12_Temperature_Analysis_Panel_Complete_View_Raw_vs_Median_FilterednVitoria_da_Conquista_September_2025_7.png}
    \caption{Comparação de filtros digitais - Temperatura Sensor 7 (Mediana)}
    \label{fig:temp_filter_median_7}
\end{figure}

\begin{figure}[H]
    \centering
    \includegraphics[width=0.95\textwidth]{signal_processing/12_Temperature_Analysis_Panel_Complete_View_Raw_vs_EWMA_FilterednVitoria_da_Conquista_September_2025_1.png}
    \caption{Comparação de filtros digitais - Temperatura Sensor 1 (EWMA)}
    \label{fig:temp_filter_ewma_1}
\end{figure}

\begin{figure}[H]
    \centering
    \includegraphics[width=0.95\textwidth]{signal_processing/12_Temperature_Analysis_Panel_Complete_View_Raw_vs_EWMA_FilterednVitoria_da_Conquista_September_2025_2.png}
    \caption{Comparação de filtros digitais - Temperatura Sensor 2 (EWMA)}
    \label{fig:temp_filter_ewma_2}
\end{figure}

\begin{figure}[H]
    \centering
    \includegraphics[width=0.95\textwidth]{signal_processing/12_Temperature_Analysis_Panel_Complete_View_Raw_vs_EWMA_FilterednVitoria_da_Conquista_September_2025_3.png}
    \caption{Comparação de filtros digitais - Temperatura Sensor 3 (EWMA)}
    \label{fig:temp_filter_ewma_3}
\end{figure}

\begin{figure}[H]
    \centering
    \includegraphics[width=0.95\textwidth]{signal_processing/12_Temperature_Analysis_Panel_Complete_View_Raw_vs_EWMA_FilterednVitoria_da_Conquista_September_2025_4.png}
    \caption{Comparação de filtros digitais - Temperatura Sensor 4 (EWMA)}
    \label{fig:temp_filter_ewma_4}
\end{figure}

\begin{figure}[H]
    \centering
    \includegraphics[width=0.95\textwidth]{signal_processing/12_Temperature_Analysis_Panel_Complete_View_Raw_vs_EWMA_FilterednVitoria_da_Conquista_September_2025_5.png}
    \caption{Comparação de filtros digitais - Temperatura Sensor 5 (EWMA)}
    \label{fig:temp_filter_ewma_5}
\end{figure}

\begin{figure}[H]
    \centering
    \includegraphics[width=0.95\textwidth]{signal_processing/12_Temperature_Analysis_Panel_Complete_View_Raw_vs_EWMA_FilterednVitoria_da_Conquista_September_2025_6.png}
    \caption{Comparação de filtros digitais - Temperatura Sensor 6 (EWMA)}
    \label{fig:temp_filter_ewma_6}
\end{figure}

\begin{figure}[H]
    \centering
    \includegraphics[width=0.95\textwidth]{signal_processing/12_Temperature_Analysis_Panel_Complete_View_Raw_vs_EWMA_FilterednVitoria_da_Conquista_September_2025_7.png}
    \caption{Comparação de filtros digitais - Temperatura Sensor 7 (EWMA)}
    \label{fig:temp_filter_ewma_7}
\end{figure}

\subsubsection{Comparação de Filtros - Umidade}

\begin{figure}[H]
    \centering
    \includegraphics[width=0.95\textwidth]{signal_processing/12_Humidity_Analysis_Panel_Complete_View_Raw_vs_FilterednVitoria_da_Conquista_September_2025_1.png}
    \caption{Comparação de filtros digitais - Umidade Sensor 1 (Média Móvel)}
    \label{fig:humid_filter_ma_1}
\end{figure}

\begin{figure}[H]
    \centering
    \includegraphics[width=0.95\textwidth]{signal_processing/12_Humidity_Analysis_Panel_Complete_View_Raw_vs_FilterednVitoria_da_Conquista_September_2025_2.png}
    \caption{Comparação de filtros digitais - Umidade Sensor 2 (Média Móvel)}
    \label{fig:humid_filter_ma_2}
\end{figure}

\begin{figure}[H]
    \centering
    \includegraphics[width=0.95\textwidth]{signal_processing/12_Humidity_Analysis_Panel_Complete_View_Raw_vs_Median_FilterednVitoria_da_Conquista_September_2025_1.png}
    \caption{Comparação de filtros digitais - Umidade Sensor 1 (Mediana)}
    \label{fig:humid_filter_median_1}
\end{figure}

\begin{figure}[H]
    \centering
    \includegraphics[width=0.95\textwidth]{signal_processing/12_Humidity_Analysis_Panel_Complete_View_Raw_vs_Median_FilterednVitoria_da_Conquista_September_2025_2.png}
    \caption{Comparação de filtros digitais - Umidade Sensor 2 (Mediana)}
    \label{fig:humid_filter_median_2}
\end{figure}

\begin{figure}[H]
    \centering
    \includegraphics[width=0.95\textwidth]{signal_processing/12_Humidity_Analysis_Panel_Complete_View_Raw_vs_EWMA_FilterednVitoria_da_Conquista_September_2025_1.png}
    \caption{Comparação de filtros digitais - Umidade Sensor 1 (EWMA)}
    \label{fig:humid_filter_ewma_1}
\end{figure}

\begin{figure}[H]
    \centering
    \includegraphics[width=0.95\textwidth]{signal_processing/12_Humidity_Analysis_Panel_Complete_View_Raw_vs_EWMA_FilterednVitoria_da_Conquista_September_2025_2.png}
    \caption{Comparação de filtros digitais - Umidade Sensor 2 (EWMA)}
    \label{fig:humid_filter_ewma_2}
\end{figure}

\subsubsection{Comparação de Filtros - Pressão}

\begin{figure}[H]
    \centering
    \includegraphics[width=0.95\textwidth]{signal_processing/12_Pressure_Analysis_Panel_Complete_View_Raw_vs_FilterednVitoria_da_Conquista_September_2025_1.png}
    \caption{Comparação de filtros digitais - Pressão Sensor 1 (Média Móvel)}
    \label{fig:press_filter_ma_1}
\end{figure}

\begin{figure}[H]
    \centering
    \includegraphics[width=0.95\textwidth]{signal_processing/12_Pressure_Analysis_Panel_Complete_View_Raw_vs_FilterednVitoria_da_Conquista_September_2025_2.png}
    \caption{Comparação de filtros digitais - Pressão Sensor 2 (Média Móvel)}
    \label{fig:press_filter_ma_2}
\end{figure}

\begin{figure}[H]
    \centering
    \includegraphics[width=0.95\textwidth]{signal_processing/12_Pressure_Analysis_Panel_Complete_View_Raw_vs_Median_FilterednVitoria_da_Conquista_September_2025_1.png}
    \caption{Comparação de filtros digitais - Pressão Sensor 1 (Mediana)}
    \label{fig:press_filter_median_1}
\end{figure}

\begin{figure}[H]
    \centering
    \includegraphics[width=0.95\textwidth]{signal_processing/12_Pressure_Analysis_Panel_Complete_View_Raw_vs_Median_FilterednVitoria_da_Conquista_September_2025_2.png}
    \caption{Comparação de filtros digitais - Pressão Sensor 2 (Mediana)}
    \label{fig:press_filter_median_2}
\end{figure}

\begin{figure}[H]
    \centering
    \includegraphics[width=0.95\textwidth]{signal_processing/12_Pressure_Analysis_Panel_Complete_View_Raw_vs_EWMA_FilterednVitoria_da_Conquista_September_2025_1.png}
    \caption{Comparação de filtros digitais - Pressão Sensor 1 (EWMA)}
    \label{fig:press_filter_ewma_1}
\end{figure}

\begin{figure}[H]
    \centering
    \includegraphics[width=0.95\textwidth]{signal_processing/12_Pressure_Analysis_Panel_Complete_View_Raw_vs_EWMA_FilterednVitoria_da_Conquista_September_2025_2.png}
    \caption{Comparação de filtros digitais - Pressão Sensor 2 (EWMA)}
    \label{fig:press_filter_ewma_2}
\end{figure}

\textbf{Resultados da filtragem digital:}
\begin{itemize}
    \item \textbf{Filtro Mediana:} Melhor supressão de ruído sem lag de fase
    \item \textbf{EWMA:} Transições suaves, preserva tendências
    \item \textbf{Média Móvel:} Trade-off entre suavização e responsividade
    \item Painéis completos com sinal bruto, filtrado, zoom e espectro de frequência
\end{itemize}

\subsection{Notebook 13: Análise de Frequência (FFT)}

\textbf{Nota:} As análises FFT foram integradas aos painéis de filtros digitais (Notebook 12), mostrando o espectro de potência de cada sinal antes e depois da filtragem.

\textbf{Resultados da análise espectral:}
\begin{itemize}
    \item \textbf{Pico primário:} Periodicidade de 23 horas (ciclo diurno)
    \item Assinatura espectral clara do ciclo de aquecimento/resfriamento diário
    \item Validação da adequação do intervalo de amostragem de 30 segundos
\end{itemize}

\section{Dataset Público}

O dataset completo foi disponibilizado publicamente no Kaggle:

\begin{itemize}
    \item \textbf{Título:} ``Vitória da Conquista Weather Data - September 2025''
    \item \textbf{URL:} \url{https://www.kaggle.com/datasets/jonassouza872/vitoria-da-conquista-weather-data-september}
    \item \textbf{Registros:} 82.430 medições
    \item \textbf{Formato:} CSV com 12 variáveis ambientais
    \item \textbf{Licença:} Acesso livre, público, para download
\end{itemize}

\section{Estrutura do Repositório}

\begin{verbatim}
PolySense-Station-/
|-- main.py                          # 226 linhas MicroPython
|-- requirements.txt                 # Dependencias Python
|-- lib/                             # Drivers MicroPython (9 arquivos)
|-- notebooks/                       # 13 notebooks de analise
|   |-- 01_exploratory_analysis.ipynb
|   |-- 02_correlation_analysis.ipynb
|   |-- 03_missing_data.ipynb
|   |-- 04_sensor_validation.ipynb
|   |-- 05_temporal_analysis.ipynb
|   |-- 06_time_series_decomposition.ipynb
|   |-- 07_anomaly_detection.ipynb
|   |-- 08_decision_tree_regression.ipynb
|   |-- 09_gmm_clustering.ipynb
|   |-- 10_kmeans_clustering.ipynb
|   |-- 11_lstm_prediction.ipynb
|   |-- 12_digital_filters.ipynb
|   |-- 13_fft_analysis.ipynb
|-- data/raw/                        # 4 arquivos CSV (4.8 MB total)
|   |-- climate_clusters_gmm.csv     # 82.430 registros
|   |-- validation_data_cleaned_BRT.csv
|   |-- validation_and_Measured_Data_cleaned_BRT_.csv
|   |-- inmet_weather_station_data_sep_2025_utc.csv
|-- images/                          # 111 visualizacoes geradas
|   |-- data_analysis/               # EDA, correlacao, validacao
|   |-- machine_learning/            # Clustering, anomalias, regressao
|   |-- signal_processing/           # Comparacoes de filtros, FFT
|-- PCB/                             # Design de PCB customizado
|-- Schematic/                       # Esquematicos de hardware
|-- README.md                        # Documentacao completa
\end{verbatim}

\section{Conclusões}

\subsection{Principais Conquistas}

\begin{enumerate}
    \item \textbf{Sistema de coleta funcional:} 100\% de sucesso ao longo de 30 dias (82.430 registros com intervalo de 30s)

    \item \textbf{Redundância de sensores eficaz:} 7 sensores de temperatura, 2 de umidade, 2 de pressão com >99.9\% de completude

    \item \textbf{Validação contra referência:} Bias médio < 2°C comparado à estação INMET oficial

    \item \textbf{Correlação excepcional:} r > 0.98 entre sensores redundantes.

    \item \textbf{Detecção de anomalias:} 3.2\% de outliers isolados com sucesso via Isolation Forest

    \item \textbf{Previsão precisa:} MAE < 1°C para previsões de temperatura 1 hora à frente com LSTM

    \item \textbf{Clustering climático:} 4 regimes distintos identificados e validados (GMM e KMeans)

    \item \textbf{Análise espectral:} Periodicidade de 24h clara confirmada via FFT

    \item \textbf{Dataset público:} 82.430 medições disponibilizadas no Kaggle para comunidade científica
\end{enumerate}

\subsection{Limitações do Estudo}

\begin{itemize}
    \item Período de coleta limitado a 30 dias (setembro 2025)
    \item Ausência de medição direta de radiação solar
    \item Granularidade temporal de 30 segundos (não captura eventos muito rápidos)
    \item Fatores externos não considerados (cobertura de nuvens detalhada, vento local)
    \item Validação limitada a uma estação do ano
\end{itemize}

\subsection{Trabalhos Futuros}

\begin{enumerate}
    \item Expandir coleta para 12 meses (análise sazonal completa)
    \item Adicionar sensor de radiação solar (piranômetro)
    \item Implementar anemômetro para velocidade e direção do vento
    \item Desenvolver PCB versão 2 com melhorias identificadas
    \item Integrar transmissão de dados via LoRaWAN
    \item Implementar edge computing com previsões LSTM em tempo real
    \item Análise comparativa entre múltiplas estações em diferentes altitudes
    \item Desenvolver API REST para acesso aos dados em tempo real
\end{enumerate}

\section{Referências}

\begin{itemize}
    \item \textbf{Microcontrolador:} Raspberry Pi Pico (RP2040), MicroPython Documentation
    \item \textbf{Estação de referência:} INMET - Instituto Nacional de Meteorologia
    \item \textbf{Dataset:} \url{https://www.kaggle.com/datasets/jonassouza872/vitoria-da-conquista-weather-data-september}
    \item \textbf{Período:} 31/08/2025 - 30/09/2025 (82.430 registros)
    \item \textbf{Bibliotecas Python:} Pandas, NumPy, Matplotlib, Seaborn, Scikit-learn, TensorFlow, SciPy
    \item \textbf{Técnicas de ML:} Isolation Forest, Decision Tree, KMeans, GMM, LSTM
    \item \textbf{Processamento de sinais:} FFT, Filtros digitais (Média móvel, Mediana, EWMA)
\end{itemize}

\end{document}
